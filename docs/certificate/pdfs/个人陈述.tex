\documentclass[cs5size]{article}
\usepackage[fontset=mac]{ctex}
\usepackage{fontspec}
\setCJKmainfont[BoldFont=STHeiti, ItalicFont=STKaiti]{STSong}
\usepackage{cite}
\bibliographystyle{unsrt}%声明参考文献格式

\newcommand{\link}[1]{\textsuperscript{\textsuperscript{\cite{#1}}}} 
\headsep 0.5 true cm \topmargin 0pt \oddsidemargin 0pt
\evensidemargin 0pt \textheight 215mm \textwidth 160mm
\renewcommand\baselinestretch{1.16}
\renewcommand\arraystretch{1.2}

\setlength\parindent{2.15em}
\newcommand{\wuhao}{\fontsize{10.5pt}{\baselineskip}\selectfont}
\renewcommand{\thefootnote}{\fnsymbol{footnote}}
\usepackage{indentfirst,amsmath,amsfonts,amssymb,amsthm,cite,multirow}
\usepackage{enumerate}

\newtheorem{theo}{\sc Theorem}
\newtheorem{lemm}{\sc Lemma}
\newtheorem{rema}{\sc Remark}
\newcommand{\tod}{\stackrel{d}{\longrightarrow}}
\newcommand{\tods}{\stackrel{d^*}{\longrightarrow}}
\def\ng{\noindent \hangindent=0.9 truecm\hangafter=1}
\def\nh{\noindent \hangindent=0.8 truecm\hangafter=1}
\def\txs{\textstyle}
\def\half{\txs{1\over 2}}
\def\be{\begin{equation}}
\def\ee{\end{equation}}
\def\nn{\nonumber}
\def\bea{\begin{eqnarray}}
\def\eea{\end{eqnarray}}
\pagenumbering{arabic}

\usepackage{hyperref}
\hypersetup{
    colorlinks=true,
    linkcolor=blue,
    filecolor=blue,
    urlcolor=blue,
    pdftitle={Overleaf Example},
%    pdfpagemode=FullScreen,
    }
\urlstyle{same}
\def\link{\hyperlink}
\def\target{\hypertarget}

\usepackage{setspace}%\singlespacing %单倍行距\onehalfspacing %1.5倍行距 \doublespacing %双倍行距
%\setlength{\baselineskip}{28pt}
\setstretch{1.2} %任意行距
\def\ng{\noindent \hangindent=0.9 truecm\hangafter=1}
\def\nh{\hangindent=0.7 truecm\hangafter=1}
\usepackage{fancyhdr,booktabs}%页眉宏包
\usepackage{graphicx,color}
\renewcommand\refname{\bf \zihao{-4}{参考文献:}}
\renewcommand{\figurename}{\kaishu\small 图}
\begin{document}
\wuhao
%\pagestyle{fancy}%页眉版式
%\pagestyle{fancy}%页眉版式
%--------------------------页数命令和公式标号命令-------------------------------------
\setcounter{page}{1} %设置起始页码
\newcounter{jie}
%%%%%%%%%%%%%%%%%%%%%%%%%%%%%%%%%%%%%%%%%%%%%%%%%%%%%%%%%%%%%%%%%%%%%%%%%%%%%%%%%%%%%%
%---------------------文章头----------------------------------------------------------
%\vspace*{0.2cm}
\begin{center}
{{\LARGE\heiti 个人陈述}\\[0.6cm]}
%{\normalsize 唐\ 洁 }\\[0.1cm]}
%{\small(广西师范大学数学与统计学院, 广西桂林, 541004)}
\end{center}
%-----------------脚注----------------------------------------------------------------
%\renewcommand{\thefootnote}{\fnsymbol{footnote}} %将脚注符号设置为fnsymbol类型,即特殊符号表示
%\footnotetext[1]{{\heiti 基本信息}:{唐洁, 1995年生, 女, 湖南人. 报名号: 1000297301. 本科专业:数学与应用数学. 研究生专业:统计学,研究方向:空间计量,经验似然.}}

%---------------------------正文------------------------------------------------------

我名字叫唐洁,95年生,湖南人,以下我将从个人学习经历、学术研究经历(含所取得的成就)以及报考动机(含研究兴趣、未来发展构想)三个方面进行陈述。

\section*{一、学习经历}

我的学习经历比较坎坷。我本科是衡阳师范学院南岳学院,一所民办本科学校,俗称“三本”。尤记得高考成绩单一出来,我两眼一黑,掉入一个深不见底的漩涡里。母亲说,分数不高也要填志愿呀,我赖在床上,用被子捂住脑袋,对母亲的催促充耳不闻,就说了一句:“听你们安排。”

高考的失利,对我来说是一个很重要的自我审视的机会。从小学到初中,那种轻松就可以取得的好成绩,高中的我拼了命也找不回曾经的一点荣光。在成绩揭晓的那一刻,我再怎么想躲藏,也无处可藏。我心中是有不甘心的,高考的成绩单宣告了我彻头彻尾的失败,但我还想再证明自己一次。

怀着一种不服输的心态,在本科的学习阶段,我每天仍保持高中的作息,早上六点起床,晚上十点半回宿舍,中午在图书馆休息,无一例外。我对自己高考的失利做了一个总结,首先,心态不对,我佯装了努力,没有做到真正的努力,比如,在别人吃早饭的时间,我腾出吃早饭的时间去看书,好像很争分夺秒的样子,可是我要看什么,达成什么目标,我是没有计划的,我的计划就是坐在那里。其次,我没有自主意识,老师上课我就听课,眼睛睁得大大的,好像很认真听讲的样子,实际上我没听进去,我应该有意识地抓住老师课堂的重点,做到真正参与其中。最后,性格特点,我性格上比较独立,遇到问题常常自我化解,与他人沟通总觉得是给别人添麻烦,高中的问题那么多,比如,三角函数有什么用?证明为什么是这样的,换一种行不行?科学家是怎么想到这个概念的?许多问题需要很深的专业背景才能解答,我自然是想不出来的,因此常常陷入死胡同,无解又无助。

在本科里,我保持着与高中同样的作息,我心态发生了变化。我与毕业的学长学姐交流,明白毕业需要什么证书,我就往那个方向冲。在入学的第一学期就报考了国家计算机二级之程序语言设计C++,当时学校并没有开课,我也没有任何计算机编程基础。我靠着自己买来的书和题库开始自学,正常上课之外的时间,我全部精力都用于做题。全情的付出换来了回报,我是我们年级第一个拿到计算机证书的人,现在回头看,这应该是很自然的事情,可在当时的我来看,那是第一次自己指挥自己,并打赢一场小战役。从这之后,我跟自己说,要么不做要么做好,选择做就要尽自己最大的努力,虽然会失败,会换来嘲笑,但是自己是真的努力了,总比假努力好,我不是天赋异禀的人,我没有靠吊儿郎当换来瞩目成绩的好命。我知道自己英语不好,每天晨起去操场大声练口语,我知道专业是第一位的,我认真听好每一堂专业课,及时回顾老师讲的东西,做习题加以巩固。为了突破自己,我考了很多证书,比如计算机三级,会计从业资格证,参加很多比赛,比如数学建模,英语演讲比赛等。一个个小目标组成了我的本科生活。

这个时期的我,有幸阅读了吴军老师的《数学之美》和《浪潮之巅》,里面用生动易懂的语言勾勒了计算机的发展以及数学在计算机领域的作用,书里回答了一个困扰我许久的问题,“除了买菜,数学有什么用?”。我感受到了数学的魅力,小小的三角函数还能刻画事物之间的相关程度,这激起了我的好奇,我希望我学的数学也能在生活中派上用场,是的,我总希望自己有点用,学的东西能用出来。我对自己未来的样子,有一个初步的设想,我要学好英语、数学和编程,成为三位一体的有用之材。

2017年大学毕业,我参加农村义务教育阶段学校特设岗位计划,成为一名乡村教师。大四那年,我想过考研,又担心家中负担重(家中三子妹,我最大),于是选择先就业沉淀,认真思考一番自己到底追求的是什么。这段工作经历,虽不是学习新知识,却是更为重要的学习经历。首先,我的心灵获得自由,我能成为一名教师有一份稳定的工作,我的母亲甚是欣慰,她不再设置种种要求,曾经的“没得商量”变成“你喜欢就去做吧”,爱恋自由,工作自由,门禁都没有了。其次,工作使我获得存在感,学生的信任,领导的赏识,给予了我很多的力量。我热爱我的工作,我希望拓展自己专业上的深度,继续升学的意愿愈加强烈。

\section*{二、学术研究经历}

2020年特岗三年服务期满,我所教授的班级也都顺利毕业,我也很幸运地考上广西师范大学。统计学专业是我喜欢的专业,这个专业贴近生活,锻炼我的诸多能力,比如数学功底,编程能力等。我总结了本科时期,优点和缺点,优点就是能执着于目标直到实现为止,缺点就是目标之间没有很大的关联,没有一个主心骨。我跟自己说,这一次,我要全力以赴不留遗憾。
研究生阶段,我的目标删繁就简,只有一个,剑指专业,一切以专业为中心,其他目标辅助专业。我剪去长发,戴上帽子,换上运动装,一切以简单方便为主,沉浸式体验学术研究。
通过钻研学术,我学到了很多重要的品质,比如:真实,事实,逻辑+细心+耐心,分享+尊重,辩证。

\smallskip \noindent \smallskip 
1. 真实

\nh {\kaishu 做人是做事的前提,做事映照做人。做论文,尤其实验数据需要还原真实,能让别人可以复刻出来,那么就不能虚荣,不能舞弊,要真实地记录。}

\smallskip \noindent \smallskip
2. 事实

\nh {\kaishu 事实,指对于这个知识,知之为知之,不知为不知。计算推导的每一步都要尊重客观规律,不能臆想不能自创,要在书中找到依据。要客观,不能主观,不能想当然。}

\smallskip \noindent \smallskip
3. 逻辑+细心+耐心

\nh {\kaishu 逻辑不仅是体现在计算上,更体现在写作上。论文其实是向别人展示你的研究成果,自然不希望糟糕的表述使成果大打折扣,如何写好论文,这需要逻辑。直截了当地向读者展示自己要解决的问题,传统方法的优缺点,自己的方法的创新点,这些需要在段落与段落之间,句子与句子之间,按照逻辑进行陈述,是个细活。}

\bigskip

\nh {\kaishu 细心这一点是我跟着秦老师学的。他很认真地阅读我写的东西并真诚地不带批评地给我建议,他仔细检查我的论文是否有语法错误,段落安排是否恰当,是否突出亮点等等。批改论文比写论文还磨人,我的老师,没有一句怨言,没有一句重话,他认真仔细让我认识到做学术该有的态度。}

\bigskip

\nh {\kaishu 耐心这一点也是跟秦老师学的。老师自己写了某个领域的一篇综述,开学的时候就发给我们看了,在我们眼里是很完整的稿件了,直接能投了。等到了期末,老师还在修改,还给我们看他修改出来的问题哪里哪里哪里,要怎么怎么怎么改。我问,这些问题是编辑社那边提出来的吗?老师说不是,他还没有投出去,是我自己检查看出来的。我们眼中,完美;老师眼中,瑕疵。我感受到什么叫慢工出精品。}

\bigskip

\nh {\kaishu 逻辑是立论文的框架,细心和耐心是论文的血肉。从学习写论文中,我懂得做事情,得先有一个轮廓,再逐渐完善,到接近完美。}

\smallskip \noindent \smallskip 
4. 分享+尊重

\nh {\kaishu 我对世界的了解真的很少,说自己是井底之蛙一点都不错。进入研究生阶段,仿佛踏入另一个世界,这个世界我从未听说过。我不知道知识原来是以一篇篇论文的形式出现,它首先出现在某个科学家的脑海,然后通过实践进行验证,写成论文发表,被伯乐识中,得到普及并扩大影响。如果这个知识到了每个人都应该知道的地步,就会写成教材,面向更广的群众。如此可见,小初高的教材那叫“通识”,换而言之,属于基本常识。这些学者们辛苦得到的东西,愿意与他人共享自己的成果,我真的觉得了不起。无私也好,功利也罢,这件事情的本身就很有意义。}

\bigskip

\nh {\kaishu 从前面的论述中也知道,写一篇论文不容易,要严谨简洁通俗,探索知识可就更不容易了,总之,不是一件好玩的事情。难怪别人说,当一名学者是反人类的,我们从远古进化而来,一直在丛林里穿梭,我们直立行走是为了更好地活动,但是当一名学者最基本的就是要坐得住呀,还要克服惰性散漫等等。尊重是我该有的态度。}

\smallskip \noindent \smallskip 
5. 辩证

\nh {\kaishu 学会辩证地对待他人科研成果,这很难且必须,这也是学术进步的原动力。已然把学者们和学者们的作品升上一个高度,你还不能像拜佛一样,盲目崇拜,那就变成了迷信,仍要以科学的眼光对待,用辩证法分析。用老师的话说,虽然这很难。}

\bigskip

除了上述提及的几点品质,还有许多优良品质是我在学者们身上看到的,比如谦和、坚持等等。研究生这一站,我收获颇丰,也取得一些小小成绩。研二正式进入研究阶段,我与老师积极沟通,共同协作,已完成3篇论文并投稿,其中1篇中文,2篇英文。研三阶段,我接到老师布置的新任务,这是一个前所未有的挑战,几乎没有现存的论文可以参考,但我们仍然取得了不错的进展,目前正在撰写新的论文,预计2篇。

\section*{ 三、报考动机}

喜欢+挑战=我的报考动机,喜欢程度$\times$应战能力=我的续航能力。

统计学的应用很广,它能与众多学科进行结合,例如医学,经济学,计算机科学等等,掌握统计学知识可以完成很多事情,比如一个贝叶斯公式就能完成垃圾邮件的过滤或者邮件分类等问题。理解统计学思想,我们可以更从容地面对生活中不确定因素,对生活更有信心,例如,当你很沮丧自己没有成功的时候,概率论会告诉你,再坚持一下,再重复几遍,就连地球出现生命的小概率事件都能在时间是无限的情况下必然出现,你还有啥好畏惧的呢。所以,我很喜欢统计学,我不满足于研究生阶段的浅尝辄止,我想在这个专业上完成最后一万二千里长征,虽然有点难。

除了对自然科学的兴趣,统计学的热爱,我的性格里还带点冒险,喜欢挑战。我对蹦极、走悬崖、攀岩等户外极限运动都很有兴趣并非常愿意一试,若我能完成一些“不可能”的事情,我会在内心给予自己很大的鼓励。因为博士难读,所以我想挑战一下。

我的研究兴趣在于,如何将一件事做好,做到极致,这一听就不可能,借用成甲一句话,“理想主义者不要求环境是完全理想的,但理想主义者不放弃对于理想的追求。”如果是讲一堂课,我会想怎么讲才能更好让听者在逻辑与视觉上有个很好的呈现,如果是一次汇报,我会想怎么描述才能用较少的语言刻画自己想说的东西,如果是我的专业,我主动补充相关知识让我的基本功更扎实。我专注于事情本身,事情本身是有意义的,那我只会想着怎么做好它。

我想深入学习统计学,中国人民大学是无数统计学子心之向往的学府,我也非常渴望能与各位老师学习,我的未来构想就是,坚定地在专业上走得更远一些,若有幸进入贵校进行学习,吾辈当倍加珍惜,加倍努力,做出成绩。




\end{document}
